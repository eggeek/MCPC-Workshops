\begin{frame}{Get lowest bit}
  \begin{minipage}{\textwidth}
  \onslide<1-> {
    The straightforward way:
    \inputminted[linenos=true, fontsize=\small, bgcolor=mygray, ]{python}{./src/lowb0.py}
  }
  \end{minipage}

  \begin{minipage}{.6\textwidth}
  \onslide<2-> {
    In competitive programming:
    \inputminted[linenos=true, fontsize=\small, bgcolor=mygray, ]{python}{./src/lowb.py}
  }
  \end{minipage}
\end{frame}

\begin{frame}{Prime Sieve}
\onslide<1-> {
  The straightforward way:
  \inputminted[linenos=true, fontsize=\scriptsize, bgcolor=mygray, ]{python}{./src/prime0.py}
}
\onslide<2-> {
  \small $2+3+2+5+\ldots \approx O(\frac{n^2}{log n})$ (\url{https://oeis.org/A088821})
}
\end{frame}

\begin{frame}{Prime Sieve}
  \onslide<1-> {
    \small More efficient way:
    \inputminted[linenos=true, fontsize=\scriptsize, bgcolor=mygray, ]{python}{./src/prime1.py}
  }
  \onslide<2-> {
    \small $\frac{n}{2} + \frac{n}{3} + \ldots + 1 \approx O(nlogn)$ (Harmonic sequence)
  }
\end{frame}

\begin{frame}{Prime Sieve}
  \begin{minipage}{\textwidth}
  \onslide<1-> {
    \small In competitive programming:
    \inputminted[linenos=true, fontsize=\scriptsize, bgcolor=mygray, ]{python}{./src/prime2.py}
  }
  \end{minipage}
  \begin{minipage}{\textwidth}
    %\setlength\parindent{0pt}
    \onslide<2-> \small Each number is only sieved by it's minimum divisor once.\\
    \onslide<3-> \small It's linear!
  \end{minipage}
\end{frame}

\begin{frame}{$A+B$}
\begin{columns}
  \begin{column}{0.48\columnwidth}
  \onslide<1-> The straightforward way:
  \inputminted[linenos=true, fontsize=\scriptsize, bgcolor=mygray, ]{python}{./src/plus0.py}
  \end{column}
  \begin{column}{0.48\columnwidth}
    \onslide<2-> In competitive programming:
    \inputminted[linenos=true, fontsize=\scriptsize, bgcolor=mygray, ]{python}{./src/plus1.py}
  \end{column}
\end{columns}

\begin{columns}
  \begin{column}{0.6\columnwidth}
  \onslide<3->
    \begin{figure}
      \centering
      \includegraphics[width=.35\textwidth]{pic/rage.jpeg}
    \end{figure}
  \end{column}
  \begin{column}{0.4\columnwidth}
    \onslide<3-> \st{Creative!}
  \end{column}
\end{columns}
\end{frame}
